%----------------------------------------------------------------------------------------
%	PACKAGES AND OTHER DOCUMENT CONFIGURATIONS
%----------------------------------------------------------------------------------------

\documentclass[paper=a4, fontsize=11pt]{scrartcl} % A4 paper and 11pt font size

\usepackage[T1]{fontenc} % Use 8-bit encoding that has 256 glyphs
%\usepackage{fourier} % Use the Adobe Utopia font for the document - comment this line to return to the LaTeX default
\usepackage[english]{babel} % English language/hyphenation
\usepackage{amsmath,amsfonts,amsthm} % Math packages

\usepackage{indentfirst} 
\usepackage{listings}
\usepackage{syntax}

\usepackage{sectsty} % Allows customizing section commands
\allsectionsfont{\centering \normalfont\scshape} % Make all sections centered, the default font and small caps
\usepackage{geometry}
\geometry{left=2cm,right=2cm,top=2.5cm,bottom=2.5cm}

\usepackage{fancyhdr} % Custom headers and footers
\pagestyle{fancyplain} % Makes all pages in the document conform to the custom headers and footers
\fancyhead{} % No page header - if you want one, create it in the same way as the footers below
\fancyfoot[L]{} % Empty left footer
\fancyfoot[C]{} % Empty center footer
\fancyfoot[R]{\thepage} % Page numbering for right footer
\renewcommand{\headrulewidth}{0pt} % Remove header underlines
\renewcommand{\footrulewidth}{0pt} % Remove footer underlines
\setlength{\headheight}{10pt} % Customize the height of the header

\numberwithin{equation}{section} % Number equations within sections (i.e. 1.1, 1.2, 2.1, 2.2 instead of 1, 2, 3, 4)
\numberwithin{figure}{section} % Number figures within sections (i.e. 1.1, 1.2, 2.1, 2.2 instead of 1, 2, 3, 4)
\numberwithin{table}{section} % Number tables within sections (i.e. 1.1, 1.2, 2.1, 2.2 instead of 1, 2, 3, 4)

\setlength\parindent{0pt} % Removes all indentation from paragraphs - comment this line for an assignment with lots of text

%----------------------------------------------------------------------------------------
%	TITLE SECTION
%----------------------------------------------------------------------------------------

\newcommand{\horrule}[1]{\rule{\linewidth}{#1}} % Create horizontal rule command with 1 argument of height

\title{	
\normalfont \normalsize 
\textsc{CS 6110 Final Report, University of Utah} \\ [25pt] % Your university, school and/or department name(s)
\horrule{0.5pt} \\[0.4cm] % Thin top horizontal rule
\huge Verifying Contracts for Dynamic Typing Language \\ % The assignment title
\horrule{2pt} \\[0.5cm] % Thick bottom horizontal rule
}

\author{Guannan Wei, Jian Lan} % Your name

%\date{\normalsize\today} % Today's date or a custom date
\begin{document}

\maketitle % Print the title

\section{Introduction}

This project presents a method to verify contract of functions in dynamic typing programming languages via abstract interpretation. \\

\textbf{Motivation}

Many popular languages such as Python, Perl, Ruby, JavaScript and Scheme are dynamic typing, which means the type of value of an identifier may change in runtime, and the type checking are performed in runtime rather than compile-time. This feature helps programmers to build prototype rapidly because programmers don't need to firstly write type annotations and make the annotations be accepted by type checker.

But dynamic typing also makes program much easier to have bugs since programmer can not catch errors until it raised in runtime. For example, as Figure \ref{fig:example} shows, the function \texttt{add1} executes a simple arithmetic operation that adds \texttt{1} to \texttt{x}. But if we pass a string argument \texttt{"3"} to \texttt{add1}, the type error \textit{only} will be raised when program runs to that call site.

\begin{figure}[h!]
\lstset{language=Lisp}
\begin{lstlisting}[frame=single]
(define (add1 x) (+ x 1)) ;Defining a function
(add1 3)                  ;OK
(add1 "3")                ;Error
\end{lstlisting} 
\caption{An example in Scheme}
\label{fig:example}
\end{figure}

Modern languages such as Racket, a dialect of Scheme provide a way to ease this problem: programmers can write contract for function, then the language's contract system will check the argument and returned value is satisfied these contracts or not in runtime(TODO: REF). The contracts are just composed of ordinary functions in the language that return a boolean value, and the way of checking contract is to straightforwardly interpret these function with runtime value, if it returns \texttt{true} then it satisfies the contract, otherwise the contract system will properly blame somewhere violates the contract.

Contract provides a way to describe refined behavior of a function, and programmers can utilize contract system and write test cases to syntactically cover the program to find bugs early, but the drawback is it may involve additional performance cost in runtime because contract checking happens on each function application.

The motivation of our project is we don't want to lose the advantages of dynamic typing, but we also want to write correct programs and catch errors as early as possible, for example, before running the program. So we present a method to check contracts statically in dynamic typing languages. \\

\textbf{Approach}

Our approach is based on abstracting abstract machine(TODO: REF), which is a technique that using abstract machine such as CESK machine (TODO: REF) to do abstract interpretation by transforming concrete semantics to abstract semantics. 

The key idea of our approach is regarding predicate (i.e. the contract) as abstract value. A predicate is an approximation of a value, it describes the bound of the value. We build an non-deterministic abstract interpreter that do computations on these abstract values, the term \textit{non-deterministic} means it will explore every possible behaviors of program's runtime. This approach based on abstract interpretation is sound, which means if the input of abstract interpretation is a sound approximation of concrete value, it's guaranteed that the result of abstract interpretation is still a sound approximation of returned concrete value. So that we could leverage abstract interpretation to check the behaviors of a function that depicted by contracts.

Section 2 describes basic setting of the language, introduces the concrete semantics and abstract semantics for A-Normal Form lambda calculus on CESK machine.

Section 3 TODO. 
Section 4 TODO.
Section 5 TODO. \\

At last, I would like to explain the reason why we change the project name. At the very beginning, we propose our project as \textit{Extending Scheme with Liquid Type}, where Liquid Type is the abbreviation of \textit{Logically Qualified Data Types}. Liquid Type is a type system based on Hindley-Milner type with predicate abstraction and could infer the refinement part of a type(TODO: REF). It's a powerful type system that can help programmers to write pre-condition and post-condition of a function, and catch errors before running the program. But later, as we go further in this project, we realized that in such context of dynamic typing functional programming languages, \textit{contract} is a more precise and conventional term to describe the pre-conditions and post-conditions. And what we are doing is actually verifying these contracts statically, i.e. without really running the program. Yet, the techniques behind the title does not change.

%----------------------------------------------------------------------------------------

\section{Abstracting Abstract Machine}

In this section, we discuss the syntax and semantics of our core language. To formally describe the concrete and abstract semantics, we use CESK machine as execution model, and present the concrete CESK machine and abstract CESK machine, respectively. Finally, we show a special continuation of abstract CESK machine to trace the argument and returned value of function application, this information will be used in contract checking.

%------------------------------------------------

\subsection{The Language}

Our small-step abstract interpreter use a small language which is a variant of call-by-value \textit{A-Normal Form} lambda calculus(TODO: REF). Traditionally A-Normal Form used is an intermediate representation of functional program as an alternative choice of Continuation-Passing Style (CPS) in compiler, it has several settings to simplify compiler as well as interpreter for our project, and more importantly, it's much more human-readable compared with CPS form. Programs in direct style can be easily transformed into A-Normal Form(TODO: REF).

The Figure \ref{fig:anf} shows the syntax of our tiny language.

\begin{figure}[h!]
\setlength{\grammarparsep}{7pt plus 1pt minus 1pt} % increase separation between rules
\setlength{\grammarindent}{8em} % increase separation between LHS/RHS 
\begin{grammar}
<var> ::= <symbol>

<label> ::= <symbol>

<integer> ::= $\cdots$ | \texttt{-1} | \texttt{0} | \texttt{1} | $\cdots$

<boolean> ::= \texttt{true} | \texttt{false}

<lambda> ::= (\texttt{lambda} [<label>] (<var>) <exp>)

<aexp> ::= <lambda>
\alt <var>
\alt <integer>
\alt <boolean>
\alt (<prim> <aexp>*)

<cexp> ::= (<aexp> <aexp>)
\alt (\texttt{if} <aexp> <exp> <exp>)
\alt (\texttt{letrec} ((<var> <aexp>)) <exp>)

<exp> ::= <aexp>
\alt <cexp>
\alt (\texttt{let} ((<var> <exp>)) <exp>)

<prim> ::= \texttt{+} | \texttt{-} | \texttt{*} | \texttt{=} | \texttt{and} | \texttt{or} | \texttt{not}
\end{grammar}
\caption{Syntax of ANF}
\label{fig:anf}
\end{figure}

Not surprisingly, we use \texttt{lambda} to form a function, and \textit{<label>} is optional and mainly used for debugging. And the language has two primitive data types: integer and boolean, as well as primitive operations on these data type are provided. ANF has three kind of expressions:

\begin{itemize}
\item Atomic expressions (\textit{<aexp>}) include primitive values, functions, and operations on primitive values. Evaluating these expressions are all trivial, that is immediately and always terminate, and no side effect.

\item Complex expressions (\textit{<cexp>}) are function applications, \texttt{if} expression and \texttt{letrec} expression. A notable thing is the operand and operator of an application must be an atomic expression, and as well as the condition of \texttt{if} expression and the right hand side of \texttt{lecrec} binding. Complex expression may not terminate, for example, an infinite recursive application.

\item Expressions (\textit{<exp>}) are either atomic expression, complex expression or a \texttt{let} expression. Unlike \texttt{letrec}, the right hand side of \texttt{let} binding can be a <exp>, so all non-trivial expressions must be bound by a \texttt{let} expression.
\end{itemize}

%------------------------------------------------

\subsection{CESK Machine}

This section we will introduce the Control, Environment, Store and Continuation (CESK) Machine(TODO: REF) and its concrete semantics for our core language.

CESK machine as an abstract machine, is a formally defined tool for describing operational semantics of programming language. As we can see from its name, CESK machine has four components that compose a \textit{state} of its execution: \textbf{Control} is the current program expression, \textbf{Environment} is a map associates variable name with address, \textbf{Store} is a simulation of memory that maps from address to value, \textbf{Continuation} represents the program stack. 

Figure \ref{fig:concretecesk} shows the concrete CESK state space.

\begin{figure}[h!]
\begin{align*}
\varsigma \in \Sigma & = Exp \times Env \times Store \times Cont \\
\rho \in Env & = Var \rightarrow Addr \\
\sigma \in Store & = Addr \rightarrow D \\
\kappa \in Cont & = \texttt{Halt} ~|~ \texttt{LetK}(v,e,\rho,\kappa) \\
d \in D & = Clo + Integer + Boolean \\
clo \in Clo & = Lam \times Env \\
a \in Addr & ~ \mbox{is an infinite set of addresses}
\end{align*}
\caption{Concrete CESK state space}
\label{fig:concretecesk}
\end{figure}

The CESK machine is a state transition based machine. The execution of program is a procedure that explores the state graph, if all the state we have already seen, then the execution terminates. Note: since for now it is a concrete machine, so the state space is infinite.

Firstly we define an inject function to form the initial state by inserting the initial expression into a state with an empty environment, an empty store and a \texttt{Halt} continuation.

$$ \mathcal{I} : Exp \rightarrow \Sigma $$
$$ \mathcal{I}(exp) = \langle exp, [], [], \texttt{Halt} \rangle $$

Next we define the state transition relations $(\longmapsto_{CESK}) \subseteq \Sigma \times \Sigma$ that maps a state to another state. \\

\textbf{Atomic Expression}\\
If the current expression is an atomic expression, which means the expression is either primitive values, functions, or operation on primitive values, and we can evaluate it directly.

$$ \langle aexp, \rho, \sigma, \kappa \rangle \longmapsto_{CESK} \langle \mathcal{A}(aexp, \rho, \sigma), \rho, \sigma, \kappa \rangle $$

where $\mathcal{A}$ is an auxiliary function to evaluate atomic expression and defined as follow:
\begin{align*}
\mathcal{A} : AExp \times Env & \times Store \rightarrow D \\
\mathcal{A}(v, \rho, \sigma) & = \sigma(\rho(v)) \\
\mathcal{A}(z, \rho, \sigma) & = z \\
\mathcal{A}(\texttt{true}, \rho, \sigma) & = \texttt{true} \\
\mathcal{A}(\texttt{false}, \rho, \sigma) & = \texttt{false} \\
\mathcal{A}(lam, \rho, \sigma) & = clo(lam, \rho)  \\
\mathcal{A}((prim, aexp_0, \cdots aexp_i), \rho, \sigma) & = \mathcal{O}(prim)(\mathcal{A}(aexp_0, \rho, \sigma), \cdots \mathcal{A}(aexp_i, \rho, \sigma))
\end{align*}

where the $\mathcal{O} : Prim \rightarrow (Value^* \rightarrow Value)$ operator is a curry function that maps a name of primitive operation to its actual operation.\\

\textbf{Function Application}\\
For function application, since the A-Normal Form already required the operator and operand must be atomic expression, so that we can reuse the $\mathcal{A}$ to evaluate them, update the environment and store and then go into the function body.
$$ \langle (aexp_0~aexp_1), \rho, \sigma, \kappa \rangle \longmapsto_{CESK} \langle body, \rho'', \sigma', \kappa \rangle $$
where 
\begin{align*}
clo((\lambda~(v)~body), \rho') & = \mathcal{A}(aexp_0, \rho, \sigma) \\
\rho'' & = \rho'[v \rightarrow a] \\
\sigma' & = \sigma[a \rightarrow \mathcal{A}(aexp_1, \rho, \sigma)]
\end{align*}

\texttt{let} \textbf{Expression}\\
\texttt{let} expression creates a binding in the body, and the right hand side and body are both expression. So we evaluate the right hand side firstly, and in the meantime create a new \texttt{LetK} continuation that save the current continuation for resuming the program in future.

$$ \langle (\texttt{let}~(lhs~rhs)~body), \rho, \sigma, \kappa \rangle \longmapsto_{CESK} \langle rhs, \rho, \sigma, \kappa' \rangle $$
where
$$ \kappa' = \texttt{LetK}(lhs, body, \rho, \kappa) $$

If the current expression in the state is already a value, and the continuation is \texttt{LetK}, which means the expression can not be reduced further and we need to resume the program from continuation.

$$ \langle v, \rho, \sigma, \texttt{LetK}(var, body, \rho', \kappa) \rangle \longmapsto_{CESK} \langle body, \rho'', \sigma', \kappa \rangle $$
where
\begin{align*}
\rho'' & = \rho'[var \rightarrow a] \\
\sigma' & = \sigma[a \rightarrow v] \\
a~&\mbox{is a fresh address}
\end{align*}

\texttt{letrec} \textbf{Expression}\\
\texttt{letrec} expression can be used for creating recursive binding, the right hand side of the binding can use name of itself. We can achieve this by firstly update the environment and then evaluate the right hand side of binding by $\mathcal{A}$, then step into the body of expression with new environment.
$$ \langle (\texttt{letrec}~(lhs~rhs)~body), \rho, \sigma, \kappa \rangle \longmapsto_{CESK} \langle body, \rho', \sigma', \kappa \rangle $$
where
\begin{align*}
\rho' & = \rho[v \rightarrow a] \\
\sigma' & = \sigma[a \rightarrow v] \\
v & = \mathcal{A}(rhs, \rho, \sigma) \\
a~&\mbox{is a fresh address}
\end{align*}

\texttt{if} \textbf{Expression}\\
The evaluation of \texttt{if} expression is straightforward: the condition of expression is an atomic expression, so we can directly use $\mathcal{A}$ to evaluate it, and then step into different branch according to its value.

$$ \langle (\texttt{if}~\text{cond thn els}), \rho, \sigma, \kappa \rangle \longmapsto_{CESK} 
\begin{cases}
\langle thn, \rho, \sigma, \kappa \rangle & \mathcal{A}(cond, \rho, \sigma) \neq \texttt{false}
\\
\langle els, \rho, \sigma, \kappa \rangle & \text{otherwise}
\end{cases}$$

%------------------------------------------------

\subsection{Abstracting CESK Machine}

In this section, we will derive a non-deterministic abstracting CESK machine for performing \textit{k}-CFA-like analysis(TODO: REF). There are several changes compared with concrete CESK machine: 

\begin{itemize}
\item the continuation is store-allocated for eliminating recursive sturcture in state space
\item the addresses are abstract, and it is an finite set; we will have two kinds of address \texttt{BindAddr} and \texttt{ContAddr} representing binding address and continuation address respectively
\item all values are abstract value, namely a type tag with a set of predicates, which will be discussed in Section 3
\item store is a mapping from address to a set of abstract values $\mathcal{P}$, the set represents all possible values for this address
\item the state includes a \textit{time} component, which encodes the execution contexts, practically it is a list of call string history.
\end{itemize}

Figure \ref{fig:abstractcesk} shows the state space of abstracting CESK machine. Notation: we use a hat on term to denote it's an abstract component, for example, $\widehat{Env}$ is abstract environment.

\begin{figure}[h!]
\begin{align*}
\varsigma \in \widehat{\Sigma} & = Exp \times \widehat{Env} \times \widehat{Store} \times \widehat{Cont} \times Time \\
\rho \in \widehat{Env} & = Var \rightarrow \widehat{Addr} \\
\sigma \in \widehat{Store} & = \widehat{Addr} \rightarrow \mathcal{P}(\hat{D}) \\
\kappa \in Cont & = \texttt{Halt} ~|~ \texttt{LetK}(v,e,\rho,\kappa) \\
a \in \widehat{Addr} & = BindAddr + ContAddr \\
BindAddr &= Var \times Time \\
ContAddr &= Exp \times Time \\
d \in \hat{D} & = \widehat{Clo} + \widehat{Integer} + \widehat{Boolean} \\
clo \in \widehat{Clo} & = Lam \times \widehat{Env}
\end{align*}
\caption{Abstract CESK state space}
\label{fig:abstractcesk}
\end{figure}

Since in abstract CESK machine the store maps an address to a set of abstract value, so the $\mathcal{A}$ will also returns a set of abstract value $\mathcal{P}(\hat{D})$. 
$$ \mathcal{A} : AExp \times Env \times Store \rightarrow \mathcal{P}(\hat{D}) $$
The detail of computation on abstract values will be discussed in Section 3.

We also use an injection function to construct initial abstract state:

$$ \mathcal{I} : Exp \rightarrow \widehat{\Sigma} $$
$$ \mathcal{I}(exp) = \langle exp, [], [], \texttt{Halt}, [] \rangle $$

Unlike the concrete CESK machine, the transition function in abstract CESK machine is from a state to a set of states(non-deterministic):

$$(\longmapsto_{\widehat{CESK}}) \subseteq \Sigma \times \Sigma^{*}$$

Here is the transition rules:

\begin{align*}
% aexp
\langle aexp, \rho, \sigma, \kappa, t \rangle \longmapsto_{\widehat{CESK}} & \langle v, \rho, \sigma, \kappa, t' \rangle \text{ where } v \in \mathcal{A}(aexp, \rho, \sigma) \\
% application
\langle (aexp_0~aexp_1), \rho, \sigma, \kappa, t \rangle \longmapsto_{\widehat{CESK}} & \langle body, \rho'', \sigma', \kappa, t' \rangle \\
\text{ where } clo((\lambda~(v)~body), \rho') & \in \mathcal{A}(aexp_0, \rho, \sigma) \\
\rho'' & = \rho'[v \rightarrow a] \\
\sigma' & = \sigma[a \rightarrow \mathcal{A}(aexp_1, \rho, \sigma)] \\
% let 
\langle (\texttt{let}~(lhs~rhs)~body), \rho, \sigma, \kappa, t \rangle \longmapsto_{\widehat{CESK}} & \langle rhs, \rho, \sigma', \kappa', t' \rangle \\
\text{ where } \kappa' & = \texttt{LetK}(lhs, body, \rho, a) \\
a & = ContAddr((\texttt{let}~(lhs~rhs)~body), t) \\
\sigma' & = \sigma[a \rightarrow \kappa] \\
% letk
\langle v, \rho, \sigma, \texttt{LetK}(var, body, \rho', a), t \rangle \longmapsto_{\widehat{CESK}} & \langle body, \rho'', \sigma', \kappa, t' \rangle  \\
\text{ where } \rho'' & = \rho'[var \rightarrow a'] \\
\sigma' & = \sigma[a' \rightarrow v] \\
a' & = BindAddr(v, t) \\
k & \in \sigma(a)
\end{align*}



\subsection{Tracing Types of Application}

%----------------------------------------------------------------------------------------

\section{Verifying Contracts}

\subsection{Overview}
The term contracts mentioned in this report has a formal name called behavioral software contracts. This concept is connected with Bertrand Meyer's design of the Eiffel programming language, and it was first described in an articles of 1986 [Meyer, Bertrand: Design by Contract, Technical Report TR-EI-12/CO, Interactive Software Engineering Inc., 1986]: behavioral software contracts are assertions that govern the flow of values across component interfaces [Meyer 1988; 1991; 1992b].
[2 footnote We use the neutral word component to refer to the pieces of a software system. Think module, class, procedure, function, or similar units of code.]

Contracts give strong guarantees about the interaction of components, while keeping the expressivity and flexibility of dynamic typing languages. However, as we metioned in Motivation part, we can't ignore two downsides: (1) contracts are usually checked dynamically (only at run-time), and they can't help us discover bugs earlier; (2) also due to the contracs run-time check, monitor contracts are expensive.[Higher-order symbolic execution for contract verification and refutation]

To solve theses two issues, we do static verfication with contracts for dynamic languages. Verify dynamic languages is more difficult than verify static languages. We can't just use the type inference method used to analysis static languages. However, if we consider contracts as abstract values and execute programs on them, the verification can be done.

Why execute programs on contracts (abstract values) can help us do the verification? Actually, contracts are used to bound values, and they contains all the given information of the boundaries of values. Execute programs on them is acutally doing the calculation of boundaries. The inputs are pre-contracts, and the outputs are post-contracts (calculated boundaries).

\subsection{The Syntax of Contract}
The Figure \ref{figc} shows the syntax of our tiny sub-language of contract in BNF.
\begin{figure}[h!]
\setlength{\grammarparsep}{7pt plus 1pt minus 1pt} % increase separation between rules
\setlength{\grammarindent}{8em} % increase separation between LHS/RHS 
\begin{grammar}
<contract-type> ::= \texttt{Int} | \texttt{Bool} | \texttt{Any} | \texttt{($\rightarrow$} <contract-type> <contract-type>\texttt{)}
\alt \texttt{(Int} <predicate> \texttt{)} | \texttt{(Bool} <predicate> \texttt{)} | \texttt{(Any \#t)}

<predicate> ::= \texttt{\#t} | \texttt{(}<pred-op> <operand> <operand> \texttt{)}

<pred-op> ::= \texttt{\textless} | \texttt{\textgreater} | \texttt{$\leq$} | \texttt{$\geq$} | \texttt{=}
\alt \texttt{!=} | \texttt{and} | \texttt{or} | \texttt{not}

<operand> ::= <predicate> | <literal> | <self> | <var>


<literal> ::= <integer> | <boolean>

<integer> ::= $\cdots$ | \texttt{-1} | \texttt{0} | \texttt{1} | $\cdots$

<boolean> ::= \texttt{true} | \texttt{false}

<self> ::= \texttt{_}

<var> ::= <symbol>
\end{grammar}
\caption{Syntax of Contract}
\label{figc}
\end{figure}

\begin{itemize}
\item \texttt{Any} means this contract type is unknown.

\item \texttt{_} represents the self value which is described by the current contract. For instance, this contract \texttt{(Int\ (\textless\ \underscore\ 3))} describes a variable which is an integer and it's less than 3.

\item If the predicate part of a contract is \texttt{\#t}, it means this value can be any concrete value of this given type.

\item You can write multiple contracts for one function.
\end{itemize}


\subsection{Operation on Contracts (Abstract Values)}
For our current work, users only provide the contracts for user defined functions, and our program will abstractly interpret all codes with the provided contracts as input abstract values.
We know the interpretation process is actually defined by a function $\delta$ that maps an operation and its input argument values to the output value. This is true no matter the input argument values are concrete or abstract. There is no essential differece and boundary between concrete and abstract interpretation. We can choose a level of abstraction, and consider one interpretation as abstract interpretation when its level of abstraction is higher than the one we choose. However, once we decide what level of abstraction is concrete interpretation, from the output values we can differentiate the non-abstract and abstract interpretation: when there are valid input values, non-abstract interpretation has a unique answer, while the output of abstract interpretation is a set of answers. The means the abstract interpretation reduces an operation nondeterministically. e.g. $\delta(sub1, 1) = 1$, $\delta\hat{}(sub1, 1) = \{Int\}$ , $\delta\hat{}(even?, \bullet) = {true, false}$. $\delta$ represents concrete interpretation, and $\delta\hat{}$ represents abstract interpretation. Here I use $\bullet$ to represent a value we no nothing about it.
We will use two example to illustrate how to operate on contract:
\begin{itemize}
\item Suppose in one step we want to get the contract of $(+ x y)$, and we know that the contract of $x$ is \texttt{(Int\ (\textless\ \underscore\ 3))} and the contract of $y$ is \texttt{(Int\ (\textless\ \underscore\ 4))} and the result of abstract interpretation is a value with the contract of \texttt{(Int\ (\textless\ \underscore\ 7))}.

\item Cases with conditionals, when we don't know the exact value of the input condition, is a little more complicated. The result contract of $(if \bullet x y)$. Assume the same contracts for $x$ and $y$ as above, the result contract is \{\texttt{(Int\ (\textless\ \underscore\ 3))}, \texttt{(Int\ (\textless\ \underscore\ 4))}\}.

\end{itemize}
 

\subsection{Check Contracts}
%% number this
We have two parts for checking contracts:
\begin{itemize}
\item Verify the user provided contracts and user defined functions are conpatible. e.g.: assume a given contract is \texttt{(: id ($\rightarrow$ Int Bool))} and a given function is \texttt{(lambda\ id\ (x)\ x)}, this function will be rejected.

\item Verify the use of the defined functions are compatible with the contracts. e.g.: assume a given function \texttt{id}, its definition is the same as that in the item above. For a dynamic typing language without any limitation, from this definition, we can say \texttt{id} is polymorphism with input and output have the same type (and value). Let's apply it:
\begin{verbatim}
                  (let ((id (lambda id (x) x)))
                    (let ((one (id 1)))
                      (let ((fls (id false)))
                        one)))
\end{verbatim}
Consider two cases: 1) The given contract is \texttt{(: id ($\rightarrow$ Int Int))}. Since the above code also applies $id$ on \texttt{Bool}, it will be rejected. 2) The given contracts are \texttt{(: id ($\rightarrow$ Int Int))} and \texttt{(: id ($\rightarrow$ Bool Bool))}. The code will be accepted. 
\end{itemize}


%----------------------------------------------------------------------------------------

\section{Implementation}

To verify our approach, we implemented a prototype verifier for ANF in Racket language.

\subsection{AAM}
Store widening.

\subsection{Contract}

%----------------------------------------------------------------------------------------

\section{Examples}

%----------------------------------------------------------------------------------------

\section{Related Work}

%----------------------------------------------------------------------------------------

\section{Future Work}

\section{Conclusion}

\end{document}
